% Template taken from https://www.sharelatex.com/templates/journals/template-for-the-journal-of-machine-learning-research-jmlr

\documentclass[twoside,11pt]{article}

% Any additional packages needed should be included after jmlr2e.
% Note that jmlr2e.sty includes epsfig, amssymb, natbib and graphicx,
% and defines many common macros, such as 'proof' and 'example'.
%
% It also sets the bibliographystyle to plainnat; for more information on
% natbib citation styles, see the natbib documentation, a copy of which
% is archived at http://www.jmlr.org/format/natbib.pdf

\usepackage{jmlr2e}
\usepackage{amsmath}

% Definitions of handy macros can go here

\newcommand{\dataset}{{\cal D}}
\newcommand{\fracpartial}[2]{\frac{\partial #1}{\partial  #2}}

% Heading arguments are {volume}{year}{pages}{submitted}{published}{author-full-names}

\jmlrheading{1}{2019}{1-10}{9/19}{9/19}{George Engel, Troy Oster, Dana Parker, Henry Soule}

% Short headings should be running head and authors last names

\ShortHeadings{CSCI 447: Project 1}{Engel, Oster, Parker, Soule}
\firstpageno{1}

\begin{document}

\title{CSCI 447: Project 1}

\author{\name George Engel \email GeoEngel.z@gmail.com \\
       \addr Department of Engineering\\
       Montana State University\\
       Bozemane, MT 59715, USA
       \AND
       \name Troy Oster \email toster1011@gmail.com \\
       \addr Department of Engineering\\
       Montana State University\\
       Bozeman, MT 59715, USA
       \AND
       \name Dana Parker \email danaharmonparker@gmail.com \\
       \addr Department of Engineering\\
       Montana State University\\
       Bozeman, MT 59715, USA
       \AND
       \name Henry Soule \email hsoule427@gmail.com \\
       \addr Department of Engineering\\
       Montana State University\\
       Bozeman, MT 59715, USA}

\editor{Engel et al.}

\maketitle

\begin{abstract}%   <- trailing '%' for backward compatibility of .sty file

A brief, one paragraph abstract summarizing the results of the experiments
% Original text:
% This paper describes the mixtures-of-trees model, a probabilistic 
% model for discrete multidimensional domains.  Mixtures-of-trees 
% generalize the probabilistic trees of \citet{chow:68}
% in a different and complementary direction to that of Bayesian networks.
% We present efficient algorithms for learning mixtures-of-trees 
% models in maximum likelihood and Bayesian frameworks. 
% We also discuss additional efficiencies that can be
% obtained when data are ``sparse,'' and we present data 
% structures and algorithms that exploit such sparseness.
% Experimental results demonstrate the performance of the 
% model for both density estimation and classification. 
% We also discuss the sense in which tree-based classifiers
% perform an implicit form of feature selection, and demonstrate
% a resulting insensitivity to irrelevant attributes.
\end{abstract}

\begin{keywords}
    Keywords, Go, Here
%   Bayesian Networks, Mixture Models, Chow-Liu Trees
\end{keywords}

\section{Introduction}

Problem statement, including hypothesis

\section{Problem Statement}

\section{The Algorithm}
Per the instructions of the assignment, we implemented the naive Bayes algorithm on the five provided datasets. Given some example $x \in X$, where $X$ is our dataset, naive Bayes predicts the correct class $c$ of $x$ by computing the probabilities of each possible classification for $x$. For class $c$, the probability is denoted as $P(c | a_1, a_2,...,a_d)$ where $a_k$ denotes one of $d$ attribute values in $x$. To compute this probability for each class $c$, the probabilities of each attribute value are computed. For each attribute value $a_k$, we compute $P(c) * \prod^d_{i=0} P(a_i | c)$, where $P(c)$ is the probability of an attribute being classified as class $c$. The predict the correct class for $x$ we compute $\underset{c \in C}{\mathrm{argmax}}$ $P(c) * \prod^d_{i=0} P(a_i | c)$.

\section{Our Approach}
% Preprocessing

% Separating and classifying data
To properly implement naive Bayes on the 5 datasets, we first needed to properly separate and classify each data set. Each data set needed to be separated by class, and then the count and probability of each attribute value for each class needed to be computed. Our approach for this was to create, for each dataset, an associative array, where each possible classifier was a key. For each key, the corresponding values were each their own associative arrays, the keys of which were all the extant attribute values among that class. The values of each key was a set storing the count and probability of each attribute value.

% Shuffling(?)

% Loss functions
To assess the performance of our class prediction algorithm, we implemented two loss functions: Precision/Recall, and Cross Entropy.
% Loss function 1: Precision Recall
The Precision/Recall loss function computes two values, known as precision and recall, to measure the performance of our class prediction algorithm. This loss function utilizes the amounts of true positive, false positive, and false negative classifications for each class. Precision and recall are are computed as follows: 
$$Precision = \frac{1}{|C|} \sum^{|C|}_{i=1} \frac{TP_i}{TP_i + FP_i}$$
$$Recall = \frac{1}{|C|} \sum^{|C|}_{i=1} \frac{TP_i}{TP_i + FN_i}$$
$TP_i$ denotes the number of true positive classifications for class $i$, $FP_i$ denotes the number of false positive classifications for class $i$, and $FN_i$ denotes the number of false negative classifications for class $i$. $C$ is the set of all possible classes. Precision can be interpreted as a measure of how accurate true positive classifications, as it computes the fraction all positive classifications for a class that were truly positive. Recall measures accuracy among the values for each class that should have been positive, as it computes the fraction of all values that truly belonged to a class that were classified as positive.


% Loss function 2: Cross Entropy
\section{Results}

% Acknowledgements should go at the end, before appendices and references

\acks{Acknowledgements go here.}

% \acks{We would like to acknowledge support for this project
% from the National Science Foundation (NSF grant IIS-9988642)
% and the Multidisciplinary Research Program of the Department
% of Defense (MURI N00014-00-1-0637). }

% Manual newpage inserted to improve layout of sample file - not
% needed in general before appendices/bibliography.

\newpage

\appendix
\section*{Appendix A.}
\label{app:theorem}

% Note: in this sample, the section number is hard-coded in. Following
% proper LaTeX conventions, it should properly be coded as a reference:

%In this appendix we prove the following theorem from
%Section~\ref{sec:textree-generalization}:

In this appendix we prove the following theorem from
Section~6.2:

\noindent
{\bf Theorem} {\it Let $u,v,w$ be discrete variables such that $v, w$ do
not co-occur with $u$ (i.e., $u\neq0\;\Rightarrow \;v=w=0$ in a given
dataset $\dataset$). Let $N_{v0},N_{w0}$ be the number of data points for
which $v=0, w=0$ respectively, and let $I_{uv},I_{uw}$ be the
respective empirical mutual information values based on the sample
$\dataset$. Then
\[
	N_{v0} \;>\; N_{w0}\;\;\Rightarrow\;\;I_{uv} \;\leq\;I_{uw}
\]
with equality only if $u$ is identically 0.} \hfill\BlackBox

\noindent
{\bf Proof}. We use the notation:
\[
P_v(i) \;=\;\frac{N_v^i}{N},\;\;\;i \neq 0;\;\;\;
P_{v0}\;\equiv\;P_v(0)\; = \;1 - \sum_{i\neq 0}P_v(i).
\]
These values represent the (empirical) probabilities of $v$
taking value $i\neq 0$ and 0 respectively.  Entropies will be denoted
by $H$. We aim to show that $\fracpartial{I_{uv}}{P_{v0}} < 0$....\\

{\noindent \em Remainder omitted in this sample. See http://www.jmlr.org/papers/ for full paper.}


\vskip 0.2in
\bibliography{sample}

\end{document}